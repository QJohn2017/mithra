\chapter{Preface to the New Version}
\label{chapter_preface}

The effort towards developing a code that offers accurate simulation of free-electron lasers (FEL) based on first-principle equation started in the framework of project AXSIS at DESY-Center for Free-Electron Laser science (CFEL).
%
The project aimed at coherent X-ray radiation through novel schemes based on inverse-Compton scattering (ICS), i.e. interaction of a relativistic electron beam with a counter-propagating laser pulse.
%
The possibility to achieve a coherent FEL radiation in a wiggling motion with undulator period as small as optical wavelength was at the time under debate.
%
Still ongoing discussions are held in the FEL and accelerator community on the difficulties and challenges in achieving FEL gain in an ICS process.
%
An analogous state was observed in projects aiming at coherent radiation in laser plasma wake-field acceleration (LPWA). 
%
A remarkable missing ingredient in all of the above discussions was a full-wave simulation tool that solves for the field and particle evolution in the FEL undulator.

In fact, many of the proposed novel schemes pursuing coherent FEL radiation violate the basic assumptions in FEL theory.
%
This, of course, does not mean that such new schemes incorporatng brilliant ideas should be abandoned.
%
However, violation of such assumptions leads to the invalidity of typical approximations that are originiated from these assumptions and typically considered in established FEL simulation tools.
%
This situation provided me the motivation to develop a full-wave simulation tool for FEL process and the outcome is presented in this manual as the software MITHRA.
%
Certainly, the software development attempts strongly benefited from discussions with a number of colleagues, which are highly appreciated here.
%
Among them, I acknwledge the discussions with Prof. Alireza Yahaghi and support from Prof. Franz K\"artner at CFEL.
%
The collaboration with Dr. PD. Andreas Adelmann and his group was also very fruitful in further enhancement of the tool.
%
Particularly, the work of Arnau Alb\'{a} in debugging the code, checking all the implemented algorithms and reviewing the software manual is highly acknowledged.
%
Eventually, my sincere gratitude to Prof. Niels Kuster for his support by providing a wonderful working environment at IT'IS foundation that was a critical factor in the latest development of MITHRA.

After the software MITHRA was fully developed, I thought it might be of limited usage for the community.
%
The main shortcoming in using the code is the long simulation times required for the investigation of various interactions.
%
Even the smallest FEL examples and simplest undulator radiation simulations require runs on massively parallel processors.
%
This implicitly shows the utmost advantage of approximations in simulation of sophisticated instruments like a FEL.
%
Notwithstanding, I gradually observed increasing interest in using the code MITHRA.
%
Currently, MITHRA is being used in projects at SLAC, PSI and DESY aiming at novel FEL concepts.
%
To meet the needs of different projects, improvement of the MITHRA software was needed and additionally new features had to be implemented.
%
In addition, using the code in new projects revealed small bugs which had to be fixed.
%
This inspiring situation motivated me to prepare a second version of the software that successes the first version presented in \cite{fallahi2018mithra}.
%
Working on different aspects of the software to meet the needs of new projects is an endless effort.
%
As a result, there still exist features that are foreseen to be implemented in the future.

A list of new features and applied changes in version 2.0 released with this manual is as the following:
%
\begin{itemize}
	\setlength{\parskip}{0pt}
	\setlength{\itemsep}{0pt plus 1pt}
	\item In the new version, user can specify the field update algorithm to be based on non-standard finite-difference or a simple finite-difference by setting the new parameter {\tt \small \em solver} in the {\tt \small \em MESH} group.
	%
	\item In the several years of using this code, I never saw a case where bunches should be initialized in the middle of a simulation. Therefore, this option is removed from the code. This means that the parameter {\tt \small \em bunch-time-start} is no more accepted in the job-file.
	%
	\item A new group in the job file is added which is named as external-field. Through this group, fields of other devices than the undulator will be added to the simulation. Currently, addition of external electromagnetic waves to the interaction is implemented.
	%
	\item Followed by the feedback from users which was inline with my own experience, the parallelization based on combined shared and distributed memory scheme (i.e. OpenMP and MPI) was not desired. Therefore, in the next versions parallelization is merely done based on distributed memory approach using MPI. Therefore, the previously existing parameter {\tt \em \small number-of-threads} is no more parsed in the job file.
	%
	\item Possibility to adjust a bunching factor phase in the bunch initialization is added through {\tt \small \em bunching-factor-phase} parameter.
	%
	\item In the new version, simulation of a Self-Amplified Spontaneous Emission (SASE) FEL is now possible. A new boolean parameter {\tt \small \em shot-noise} is added. When it is set to true, a shot noise is calculated based on real number of electrons and subsequently introduced to the bunch. The implementation algorithm for the shot-noise is also added to the manual.
	%
	\item In the previous version, the {\tt \small \em bunch-initialization} subgroup could be repeated to initialize multiple bunches in a single simulation. While this feature is kept in the new version, the {\tt \small \em bunch-initialization} subgroup now accepts arbitrary number of {\tt \small \em position} vectors. As a result, at each position, determined by the position vector, the bunch is initialized. This feature is useful in initializing an array of bunches to be injected into the undulator.
	%
	\item Because of an application of the code, a new field type is added to the code named as confined-plane-wave. This is fundamentally similar to plane-wave that is confined to an elliptical region determined by the two radius parameters.
	%
	\item The names {\tt \em \small rayleigh-radius-parallel} and {\tt \em \small rayleigh-radius-perpendicular} are changed to {\tt \em \small radius-parallel} and {\tt \em \small radius-perpendicular}.
	%
	\item The parameter {\tt \em \small resolution} in the field-sampling category as well as the radiation-power subgroup is changed to {\tt \em \small number-of\-points} which is more meaningful. Similarly, in the radiation-power subgroup, the {\tt \em \small normalized-frequency-resolution} is changed to the {\tt \em \small number-of-frequency-points}. With these changes in parameter names, the definitions of the given values are correspondingly changed.
	%
	\item In the new version, the possibility to save 2D visualization data over Cartesian planes is added. In the field-visualization subgroup two parameters {\tt \em \small type} and {\tt \em \small plane} are added, which determine the type of the visualization (2D in-plane or 3D all-domain) and plane of the 2D data ($xy$, $xz$, or $yz$) respectively. Moreover, the field-visualization subgroup can be repeated in order to obtain different visualizations of the radiated fields.
	%
	\item Several changes are applied to the undulator part. First, different undulator types are now introduced as subgroups in the undulator section. In the new version, a subgroup named {\em \tt \small static-undulator-array} is added that defines an array of undulators with or without the tapering of the undulator parameter. For detailed description on how undulator arrays are introduced to the code, the user interface chapter can be studied. In addition, the undulator group is now repeatable, meaning that several undulators with different types can be given and superposed in a single FEL simulation.
	%
	\item Possibility to visualize the radiated power in front of the bunch over a plane perpendicular to the undulator axis is an added feature to the software. This feature is added through th subgroup {\tt \em \small power-visualization}.
	%
	\item Besides this manual, a new reference card is prepared in addition to the chapter on user interface. The content of this reference card can be used as a cheat sheet when using MITHRA. The reference card is available both separately and as a chapter in this manual. 
\end{itemize}
%
In future, adding the following aspects to the software are planned:
%
\begin{itemize}
	\setlength{\parskip}{0pt}
	\setlength{\itemsep}{0pt plus 1pt}
	\item Adding the support for computation on GPU cards
	%
	\item Adding the possibility of considering slow-wave approximation in time, space and both to obtain a fast computation with the cost of less accuracy
	%
	\item Computing the bunching factor of the bunch as an output parameter. Currently, it can be extracted by saving the bunch profile with a certain rhythm and performing post-processing separately after the simulation.
	%
	\item Computing the total radiated energy as an output parameter. Currently, it can be extracted by sampling the radiated power and performing a time-dependent integral over the radiated power.
	%
	\item Implementing a far-field transformation technique to more accurately estimate the radiated power. This will avoid the problem of power underestimation due to limited area of the power-sampling plane
	 in front of the bunch.
	%
	\item Implementing UPML boundary condition to minimize the computational domain for FEL calculations
	%
	\item Implementing quadrupole lattices in the region between undulator modules in an undulator array. These quadrupoles will be implemented as an external field subgroup. 
\end{itemize}
%
Moreover, the previously presented examples are all analyzed again with the new version and new results are illustrated in this manual.
%
In some occasions, we observed small changes compared to the old results, which are believed to happen after the removal of bugs in the previous release.
%
I plan to update the list of examples with the new projects where MITHRA is being used.
%
However, this task can be done only after the ongoing projects are closed and the results are disseminated.
%
Owing to my dedication to develop open-source softwares, I have placed the code in github for any interested user to download the code and work with it.
%
The source codes are available under the link \href{https://github.com/aryafallahi/mithra}{https://github.com/aryafallahi/mithra}.
%
Eventually, I welcome any feedback from users of the code which will be an indispensable help for further improving the software performance.
%
Besides, I appreciate if the users cite my article about the code \cite{fallahi2018mithra} in publications of the projects in which MITHRA is used.