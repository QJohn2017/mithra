\documentclass[a4paper]{article}

\usepackage[margin=2cm]{geometry}
\usepackage[english]{babel}
\usepackage{fancyhdr}
\usepackage{fancyvrb}
\usepackage{graphicx}
\usepackage{url}
\usepackage{epic}
\usepackage{eepic}
\usepackage{makeidx}
\usepackage{array}
\usepackage{times}
\usepackage{amsmath}
\usepackage{esint}
\usepackage{amssymb}
\usepackage{fancyvrb}
\usepackage[usenames,dvipsnames]{color}
\usepackage{framed}
\usepackage{multirow}
\usepackage{enumerate}
\usepackage[round,sort]{natbib}
\usepackage{lscape}
\usepackage{alltt,xcolor}
\usepackage{tabto}

\begin{document}
%
In the following, a general format for the input file of MITHRA is presented. The red icons or groups can be repeated in the text. 

\begin{Verbatim}[fontsize=\footnotesize, tabsize=2, fontfamily=courier,	fontseries=b, commandchars=\\\{\}]
MESH
\{
	length-scale							= < double | METER | DECIMETER | CENTIMETER | MILLIMETER | MICROMETER | 
																NANOMETER | ANGSTROM
	time-scale								= < double | SECOND | MILLISECOND | MICROSECOND | NANOSECOND | 
																PICOSECOND | FEMTOSECOND | ATTOSECOND >
	mesh-lengths							= < ( double, double, double ) >
	mesh-resolution		 			= < ( double, double, double ) >
	mesh-center				 			= < ( double, double, double ) >
	total-time								= < double >
	bunch-time-step		 			= < double >
	bunch-time-start  				= < double >
	mesh-truncation-order 		= < 1 | 2 >
	space-charge  						= < true | false >
\}

BUNCH
\{
\textcolor{red}{bunch-initialization}
	\textcolor{red}{\{}
		type  									= < manual | ellipsoid | 3D-crystal | file >
		distribution  					= < uniform | gaussian >
		charge  								= < double >
		number-of-particles  	 = < int >
		gamma  								 = < double >
		beta  									= < double >
		direction  = < ( double, double, double ) >
		position  = < ( double, double, double ) >
		sigma-position  = < ( double, double, double ) >
		sigma-momentum  = < ( double, double, double ) >
		transverse-truncation  = < double >
		longitudinal-truncation  = < double >
		bunching-factor  = < double between zero and one >
	\textcolor{red}{\}}

bunch-sampling
\{
sample  = < true | false >
directory  = < address according to UNIX convention >
base-name  = < name of the file >
rhythm  = < double >
\}

bunch-visualization
\{
sample  = < true | false >
directory  = < address according to UNIX convention >
base-name  = < name of the file >
rhythm  = < double >
\}

bunch-profile
\{
sample  = < true | false >
directory  = < address according to UNIX convention >
base-name  = < name of the file >
time  = < double >
rhythm  = < double >
\}
\}

FIELD
\{
field-initialization
\{
type  = < plane-wave | confined-plane-wave | gaussian-beam >
position  = < ( double , double , double ) >
direction  = < ( double , double , double ) >
polarization  = < ( double , double , double ) >
radius-parallel  = < double >
radius-perpendicular  = < double >
signal-type  = < neumann | gaussian | secant-hyperbolic | flat-top >
strength-parameter  = < double >
offset  = < double >
variance  = < double >
wavelength  = < double >
CEP  = < double >
\}

field-sampling
\{
sample  = < true | false >
type  = < over-line | at-point >
\textcolor{red}{field  = < Ex | Ey | Ez | Bx | By | Bz | Ax | Ay | Az | Jx | Jy 
\tabto{6.3cm}| Jz | F | Q >}
directory  = < address according to UNIX convention >
base-name  = < name of the file >
rhythm  = < double >
\textcolor{red}{position  = < ( double , double , double ) >}
line-begin  = < ( double , double , double ) >
line-end  = < ( double , double , double ) >
resolution  = < double >
\}

\textcolor{red}{field-visualization}
\textcolor{red}{\{}
sample  = < true | false >
\textcolor{red}{field  = < Ex | Ey | Ez | Bx | By | Bz | Ax | Ay | Az | Jx | Jy
\tabto{6.3cm}| Jz | F | Q >}
directory  = < address according to UNIX convention >
base-name  = < name of the file >
rhythm  = < double >
\textcolor{red}{\}}

field-profile
\{
sample  = < true | false >
\textcolor{red}{field  = < Ex | Ey | Ez | Bx | By | Bz | Ax | Ay | Az | Jx | Jy 
\tabto{6.3cm}| Jz | F | Q >}
directory  = < address according to UNIX convention >
base-name  = < name of the file >
rhythm  = < double >
\textcolor{red}{time  = < double >}
\}
\}

UNDULATOR
\{
\textcolor{red}{static-undulator}
\textcolor{red}{\{}
undulator-parameter  = < double >
period  = < double >
length  = < int >
polarization-angle  = < double >
offset  = < double >
\textcolor{red}{\}}

\textcolor{red}{static-undulator-array}
\textcolor{red}{\{}
undulator-parameter  = < double >
period  = < double >
length  = < int >
polarization-angle  = < double >
gap  = < double >
number  = < int >
tapering-parameter  = < double > 
\textcolor{red}{\}}

\textcolor{red}{optical-undulator}
\textcolor{red}{\{}
beam-type  = < plane-wave | confined-plane-wave | gaussian-beam >
position  = < ( double , double , double ) >
direction  = < ( double , double , double ) >
polarization  = < ( double , double , double ) >
radius-parallel  = < double >
radius-perpendicular  = < double >
signal-type  = < neumann | gaussian | secant-hyperbolic | flat-top >
strength-parameter  = < double >
offset  = < double >
variance  = < double >
wavelength  = < double >
CEP  = < double >
\textcolor{red}{\}}
\}

EXTERNAL-FIELD
\{
\textcolor{red}{electromagnetic-wave}
\textcolor{red}{\{}
type  = < plane-wave | confined-plane-wave | gaussian-beam >
position  = < ( double , double , double ) >
direction  = < ( double , double , double ) >
polarization  = < ( double , double , double ) >
radius-parallel  = < double >
radius-perpendicular  = < double >
signal-type  = < neumann | gaussian | secant-hyperbolic | flat-top >
strength-parameter  = < double >
offset  = < double >
variance  = < double >
wavelength  = < double >
CEP  = < double >
\textcolor{red}{\}}
\}

FEL-OUTPUT
\{
\textcolor{red}{radiation-power}
\textcolor{red}{\{}
sample  = < false | true >
type  = < at-point | over-line >
directory  = < address according to UNIX convention >
base-name  = < name of the file >
\textcolor{red}{plane-position  = < double >}
line-begin  = < double >
line-end  = < double >
resolution  = < double >
\textcolor{red}{normalized-frequency  = < double >}
minimum-normalized-frequency \tabto{8cm} = < double >
maximum-normalized-frequency \tabto{8cm} = < double >
normalized-frequency-resolution \tabto{8cm} = < double >
\textcolor{red}{\}}
\}
\end{Verbatim}
\end{document}