\chapter{Reference Card}
\label{chapter_refcard}

In the following, a general format for the input file of MITHRA is presented. The red icons or groups can be repeated in the text. \emph{int} stands for an integer number, \emph{real} represents a real value, and \emph{string} denotes a string of characters. The reference directory in the path locations is the path where the simulation is started. In other words, ``.\,/\," points to the location where the project is called.

\begin{multicols}{2}
\setlength{\columnseprule}{0.1pt}

\begin{Verbatim}[fontsize=\footnotesize, tabsize=2, fontfamily=courier,	fontseries=b, commandchars=\\\{\}]
MESH
\{
	length-scale							= < real | 
																METER | 
																DECIMETER | 
																CENTIMETER | 
																MILLIMETER | 
																MICROMETER |
																NANOMETER | 
																ANGSTROM >
	time-scale								= < real | 
																SECOND | 
																MILLISECOND | 
																MICROSECOND | 
																NANOSECOND | 
																PICOSECOND | 
																FEMTOSECOND | 
																ATTOSECOND >
	mesh-lengths							= < ( real, real, real ) >
	mesh-resolution		 			= < ( real, real, real ) >
	mesh-center				 			= < ( real, real, real ) >
	total-time								= < real >
	bunch-time-step		 			= < real >
	mesh-truncation-order 		= < 1 | 2 >
	space-charge  						= < true | false >
	solver										= < NSFD | FD >
\}

BUNCH
\{
	\textcolor{red}{bunch-initialization}
	\textcolor{red}{\{}
		type  									= < manual | 
																ellipsoid | 
																3D-crystal | 
																file >
		distribution  					= < uniform | gaussian >
		charge  								= < real >
		number-of-particles  	 = < int >
		gamma  								 = < real >
		beta  									= < real >
		direction  						 = < ( real, real, real ) >
		\textcolor{red}{position  							= < ( real, real, real ) >}
		sigma-position  				= < ( real, real, real ) >
		sigma-momentum  				= < ( real, real, real ) >
		numbers								 = < ( int, int, int ) >
		lattice-constants			 = < ( real, real, real ) >
		transverse-truncation   = < real >
		longitudinal-truncation = < real >
		bunching-factor  			 = < real between 0 and 1 >
		bunching-factor-phase	 = < real >
		shot-noise  						= < true | false >
	\textcolor{red}{\}}

bunch-sampling
	\{
		sample  								= < true | false >
		directory  						 = < /path/to/location >
		base-name  						 = < string >
		rhythm  								= < real >
	\}

	bunch-visualization
	\{
		sample  								= < true | false >
		directory  						 = < /path/to/location >
		base-name  						 = < string >
		rhythm  								= < real >
	\}

	bunch-profile
	\{
		sample  								= < true | false >
		directory  						 = < /path/to/location >
		base-name  						 = < string >
		\textcolor{red}{time  									= < real >}
		rhythm  								= < real >
	\}
\}

FIELD
\{
	field-initialization
	\{
		type  									= < plane-wave | 
																confined-plane-wave | 
																gaussian-beam >
		position  							= < ( real, real, real ) >
		direction  						 = < ( real, real, real ) >
		polarization  					= < ( real, real, real ) >
		radius-parallel  			 = < real >
		radius-perpendicular  	= < real >
		signal-type  					 = < neumann | gaussian | 
		secant-hyperbolic | 
		flat-top >
		strength-parameter  		= < real >
		offset  								= < real >
		pulse-length  					= < real >
		wavelength  						= < real >
		CEP  									 = < real >
	\}

field-sampling
	\{
		sample  								= < true | false >
		type  									= < over-line | at-point >
		\textcolor{red}{field  								 = < Ex | Ey | Ez |}
		\textcolor{red}{														Bx | By | Bz |}
		\textcolor{red}{														Ax | Ay | Az |}
		\textcolor{red}{														Jx | Jy | Jz |}
		\textcolor{red}{														F  | Q >}
		directory  						 = < /path/to/location >
		base-name  						 = < string >
		rhythm  								= < real >
		\textcolor{red}{position  							= < ( real, real, real ) >}
		line-begin  						= < ( real, real, real ) >
		line-end  							= < ( real, real, real ) >
		number-of-points  			= < int >
	\}

\textcolor{red}{field-visualization}
	\textcolor{red}{\{}
		sample  								= < true | false >
		type										= < in-plane | 
																all-domain >
		plane									 = < xy | yz | xz >
		position  							= < ( real, real, real ) >
		\textcolor{red}{field  								 = < Ex | Ey | Ez |}
		\textcolor{red}{														Bx | By | Bz |}
		\textcolor{red}{														Ax | Ay | Az |}
		\textcolor{red}{														Jx | Jy | Jz |}
		\textcolor{red}{														F  | Q >}
		directory  						 = < /path/to/location >
		base-name  						 = < string >
		rhythm  								= < real >
	\textcolor{red}{\}}

field-profile
	\{
		sample  								= < true | false >
		\textcolor{red}{field  								 = < Ex | Ey | Ez |}
		\textcolor{red}{														Bx | By | Bz |}
		\textcolor{red}{														Ax | Ay | Az |}
		\textcolor{red}{														Jx | Jy | Jz |}
		\textcolor{red}{														F  | Q >}
		directory  						 = < /path/to/location >
		base-name  						 = < string >
		rhythm  								= < real >
		\textcolor{red}{time  									= < real >}
	\}
\}

UNDULATOR
\{
	\textcolor{red}{static-undulator}
	\textcolor{red}{\{}
		undulator-parameter  	 = < real >
		period  								= < real >
		length  								= < int >
		polarization-angle  		= < real >
		offset  								= < real >
		distance-to-bunch-head  = < real >
	\textcolor{red}{\}}

\textcolor{red}{static-undulator-array}
	\textcolor{red}{\{}
		undulator-parameter  	 = < real >
		period  								= < real >
		length  								= < int >
		polarization-angle  		= < real >
		gap  									 = < real >
		number  								= < int >
		tapering-parameter  		= < real >
		distance-to-bunch-head  = < real >
	\textcolor{red}{\}}

	\textcolor{red}{optical-undulator}
	\textcolor{red}{\{}
		beam-type  						 = < plane-wave | 
																confined-plane-wave | 
																gaussian-beam >
		position  							= < ( real, real, real ) >
		direction  						 = < ( real, real, real ) >
		polarization  					= < ( real, real, real ) >
		radius-parallel  			 = < real >
		radius-perpendicular  	= < real >
		signal-type  					 = < neumann | gaussian | 
																secant-hyperbolic | 
																flat-top >
		strength-parameter  		= < real >
		offset  								= < real >
		pulse-length  					= < real >
		wavelength  						= < real >
		CEP  									 = < real >
		distance-to-bunch-head  = < real >
	\textcolor{red}{\}}
\}

EXTERNAL-FIELD
\{
	\textcolor{red}{electromagnetic-wave}
	\textcolor{red}{\{}
		beam-type  						 = < plane-wave | 
																confined-plane-wave | 
																gaussian-beam >
		position  							= < ( real, real, real ) >
		direction  						 = < ( real, real, real ) >
		polarization  					= < ( real, real, real ) >
		radius-parallel  			 = < real >
		radius-perpendicular  	= < real >
		signal-type  					 = < neumann | gaussian | 
																secant-hyperbolic | 
																flat-top >
		strength-parameter  		= < real >
		offset  								= < real >
		pulse-length  					= < real >
		wavelength  						= < real >
		CEP  									 = < real >
	\textcolor{red}{\}}
\}

FEL-OUTPUT
\{
	\textcolor{red}{radiation-power}
	\textcolor{red}{\{}
		sample  								= < false | true >
		type  									= < at-point | over-line >
		directory  						 = < /path/to/location >
		base-name  						 = < string >
		\textcolor{red}{plane-position  				= < real >}
		line-begin  						= < real >
		line-end  							= < real >
		number-of-points  			= < int >
		\textcolor{red}{normalized-frequency  	= < real >}
		minimum-normalized-frequency 	= < real >
		maximum-normalized-frequency 	= < real >
		number-of-frequency-points		 = < int >
	\textcolor{red}{\}}
	
	\textcolor{red}{power-visualization}
	\textcolor{red}{\{}
		sample  								= < false | true >
		directory  						 = < /path/to/location >
		base-name  						 = < string >
		plane-position  				= < real >
		normalized-frequency  	= < real >
		rhythm		 						 = < real >
	\textcolor{red}{\}}

	\textcolor{red}{bunch-profile-lab-frame}
	\textcolor{red}{\{}
		sample  								= < false | true >
		directory  						 = < /path/to/location >
		base-name  						 = < string >
		\textcolor{red}{position								= < real >}
		rhythm									= < real >
	\textcolor{red}{\}}
\}
\end{Verbatim}
\end{multicols}