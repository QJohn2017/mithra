\begin{appendices}
\chapter{Job files}
\label{job_files}

\section{Example 1: Infrared FEL}
\label{job_file_1}

\begin{snugshade}
\begin{Verbatim}[fontsize=\footnotesize, tabsize=4, fontfamily=courier, fontseries=b, commandchars=\\\{\}, obeytabs]
MESH
\{
	length-scale					= MICROMETER
	time-scale						= PICOSECOND
	mesh-lengths					= ( 3200,  3200.0,    280.0)
	mesh-resolution					= ( 50.0,    50.0,      0.1)
	mesh-center						= ( 0.0,      0.0,      0.0)
	total-time						= 30000
	bunch-time-step					= 1.6
	mesh-truncation-order			= 2
	space-charge					= false
	solver							= NSFD
\}
	
BUNCH
\{
	bunch-initialization
	\{
		type						= ellipsoid
		distribution				= uniform
		charge						= 1.846e8
		number-of-particles			= 131072
		gamma						= 100.41
		direction					= (    0.0,     0.0,       1.0)
		position					= (    0.0,     0.0,       0.0)
		sigma-position				= (  260.0,   260.0,     50.25)
		sigma-momentum				= ( 1.0e-8,  1.0e-8, 100.41e-4)
		transverse-truncation		= 1040.0
		longitudinal-truncation		= 90.0
		bunching-factor				= 0.01
	\}
\}
	
FIELD
\{
	field-sampling
	\{
		sample						= true
		type						= at-point
		field						= Ex
		field						= Ey
		field						= Ez
		directory					= ./
		base-name					= field-sampling/field
		rhythm						= 3.2
		position					= (0.0, 0.0, 110.0)
	\}
\}
	
UNDULATOR
\{
	static-undulator
	\{
		undulator-parameter			= 1.417
		period						= 3.0e4
		length						= 300
		polarization-angle			= 0.0
	\}
\}
	
FEL-OUTPUT
\{
	radiation-power
	\{
		sample						= true
		type						= at-point
		directory					= ./
		base-name					= power-sampling/power
		plane-position				= 110.0
		normalized-frequency		= 1.00
	\}
\}
\end{Verbatim}
\end{snugshade}

\section{Example 2: Seeded UV FEL}
\label{job_file_2}

\begin{snugshade}
\begin{Verbatim}[fontsize=\footnotesize, tabsize=4, fontfamily=courier, fontseries=b, commandchars=\\\{\}, obeytabs]
MESH
\{
	length-scale					= MICROMETER
	time-scale						= PICOSECOND
	mesh-lengths 					= ( 2500.0, 2500.0, 165.0  )
	mesh-resolution 				= (   30.0,   30.0,   0.02 )
	mesh-center 					= (    0.0,    0.0,   0.0  )
	total-time 						= 50000
	bunch-time-step 				= 1.6
	mesh-truncation-order 			= 2
	space-charge 					= false
	solver							= NSFD
\}
	
BUNCH
\{
	bunch-initialization
	\{
		type						= ellipsoid
		distribution				= uniform
		charge						= 3.4332e8
		number-of-particles			= 4194304
		gamma						= 391.36
		direction					= ( 0.0,  0.0,  1.0 )
		position					= ( 0.0,  0.0,  0.0 )
		sigma-position				= ( 95.3, 95.3, 75.0)
		sigma-momentum				= ( 0.0105, 0.0105, 391.36e-4)
		transverse-truncation		= 400.0
		longitudinal-truncation		= 78.0
		bunching-factor				= 0.0
	\}
	
	bunch-visualization
	\{
		sample						= true
		directory					= /cluster/scratch/afallahi/
		base-name					= bunch-visualization-seeded/bunch
		rhythm						= 500
	\}
\}
	
FIELD
\{
	field-initialization
	\{
		type						= gaussian-beam
		position					= ( 0.0, 0.0, 700000)
		direction					= ( 0.0, 0.0, 1.0)
		polarization				= ( 0.0, 1.0, 0.0)
		radius-parallel				= 183.74
		radius-perpendicular		= 183.74
		strength-parameter			= 9.857e-7
		signal-type					= gaussian
		offset						= 700000.0 #not really needed
		pulse-length				= 1.0e12
		wavelength					= 0.265187
		CEP							= 0.0
	\}
\}
	
UNDULATOR
\{
	static-undulator
	\{
		undulator-parameter			= 1.95
		period						= 2.8e4
		length						= 535
		polarization-angle			= 0.0
		offset						= 0.0
	\}
\}
	
FEL-OUTPUT
\{
	radiation-power
	\{
		sample						= true
		type						= at-point
		directory					= ./
		base-name					= power-sampling/power
		plane-position				= 78.0
		normalized-frequency		= 1.00
	\}
	
	bunch-profile-lab-frame
	\{
		sample						= true
		directory					= ./
		base-name					= bunch-profile-lab-frame/profile
		position					=  0.0e6
		position					=  2.0e6
		position					=  4.0e6
		position					=  6.0e6
		position					=  8.0e6
		position					= 10.0e6
		position					= 12.0e6
	\}
\}
\end{Verbatim}
\end{snugshade}

\section{Example 3: Infrared FEL with Optical Undulator}
\label{job_file_3}

\begin{snugshade}
\begin{Verbatim}[fontsize=\footnotesize, tabsize=4, fontfamily=courier, fontseries=b, commandchars=\\\{\}, obeytabs]
MESH
\{
	length-scale					= MICROMETER
	time-scale						= PICOSECOND
	mesh-lengths					= ( 3200,  3200.0,    280.0)
	mesh-resolution					= ( 50.0,    50.0,      0.1)
	mesh-center						= ( 0.0,      0.0,      0.0)
	total-time						= 30000
	bunch-time-step					= 1.6
	mesh-truncation-order			= 2
	space-charge					= false
	solver							= NSFD
\}
	
BUNCH
\{
	bunch-initialization
	\{
		type						= ellipsoid
		distribution				= uniform
		charge						= 1.846e8
		number-of-particles			= 131072
		gamma						= 100.41
		direction					= (    0.0,     0.0,       1.0)
		position					= (    0.0,     0.0,       0.0)
		sigma-position				= (  260.0,   260.0,     50.25)
		sigma-momentum				= ( 1.0e-8,  1.0e-8, 100.41e-4)
		transverse-truncation		= 1040.0
		longitudinal-truncation		= 90.0
		bunching-factor				= 0.01
	\}
\}
	
UNDULATOR
\{
	optical-undulator
	\{
		beam-type					= plane-wave
		position					= ( 0.0, 0.0, 0.0 )
		direction					= ( 0.0, 0.0,-1.0 )
		polarization				= ( 0.0, 1.0, 0.0 )
		strength-parameter			= 1.417
		signal-type					= flat-top
		wavelength					= 6.0e4
		pulse-length				= 18.0e6
		offset						=  9.0e6
		CEP							= 0.0
	\}
\}
	
FEL-OUTPUT
\{
	radiation-power
	\{
		sample						= true
		type						= at-point
		directory					= ./
		base-name					= power-sampling/power
		plane-position				= 110.0
		normalized-frequency		= 1.00
	\}
\}
\end{Verbatim}
\end{snugshade}

\section{Example 3: Inverse Compton Scattering}
\label{job_file_4}

\begin{snugshade}
\begin{Verbatim}[fontsize=\footnotesize, tabsize=4, fontfamily=courier, fontseries=b, commandchars=\\\{\}, obeytabs]
MESH
\{
	length-scale                     = NANOMETER
	time-scale                       = ATTOSECOND
	mesh-lengths                     = ( 2000.0, 2000.0, 165.0 )
	mesh-resolution                  = (    5.0,    5.0,   0.05)
	mesh-center                      = (    0.0,    0.0,   0.0 )
	total-time                       = 1300000
	bunch-time-step                  = 20.0
	mesh-truncation-order            = 2
	space-charge                     = false
	solver							= NSFD
\}

BUNCH
\{
	bunch-initialization
	\{
		type                           = ellipsoid
		distribution                   = uniform
		charge                         = 2800
		number-of-particles            = 2800
		gamma                          = 30.0
		direction                      = (  0.0,   0.0,   1.0)
		position                       = (  0.0,   0.0,   0.0)
		sigma-position                 = ( 60.0,  60.0,  72.0)
		sigma-momentum                 = ( 0.03,  0.03,  0.03)
		transverse-truncation          = 240.0
		longitudinal-truncation        = 77.0
		bunching-factor                = 0.0
		shot-noise						= true
	\}
\}

FIELD
\{
	field-sampling
	\{
		sample                         = true
		type                           = at-point
		field                          = Ex
		field                          = Ey
		field                          = Ez
		directory                      = ./
		base-name                      = field-sampling/field
		rhythm                         = 3.2
		position                       = (0.0, 0.0, 80.0)
	\}
\}

UNDULATOR
\{
	optical-undulator
	\{
		beam-type                        = plane-wave
		position                         = ( 0.0, 0.0, 0.0 )
		direction                        = ( 0.0, 0.0,-1.0 )
		polarization                     = ( 0.0, 1.0, 0.0 )
		strength-parameter               = 0.5
		signal-type                      = flat-top
		wavelength                       = 1.0e3
		pulse-length                    = 1.2e6
		offset 								= 6.0e6
		CEP                              = 0.0
	\}
\}

FEL-OUTPUT
\{
	radiation-power
	\{
		sample                         = true
		type                           = at-point
		directory                      = ./
		base-name                      = power-sampling/power
		plane-position            = 82
		normalized-frequency           = 1.0
		normalized-frequency           = 2.0
		normalized-frequency           = 3.0
	\}
	
	bunch-profile-lab-frame
	\{
		sample						= true
		directory					= ./
		base-name					= bunch-profile-lab-frame/profile
		rhythm						=  1.0e3
	\}
\}
\end{Verbatim}
\end{snugshade}

\end{appendices}